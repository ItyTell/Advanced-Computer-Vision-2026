\documentclass[12pt,a4paper]{article}
\usepackage[utf8]{inputenc}
\usepackage[ukrainian]{babel}
\usepackage{amsmath}
\usepackage{graphicx}
\usepackage{listings}
\usepackage{xcolor}
\usepackage{geometry}

\geometry{margin=2cm}

\definecolor{codegreen}{rgb}{0,0.6,0}
\definecolor{codegray}{rgb}{0.5,0.5,0.5}
\definecolor{codepurple}{rgb}{0.58,0,0.82}
\definecolor{backcolour}{rgb}{0.95,0.95,0.92}

\lstset{
    language=Python,
    backgroundcolor=\color{backcolour},
    commentstyle=\color{codegreen},
    keywordstyle=\color{magenta},
    numberstyle=\tiny\color{codegray},
    stringstyle=\color{codepurple},
    basicstyle=\ttfamily\small,
    breaklines=true,
    showstringspaces=false
}

\begin{titlepage}
    \centering
    \vspace*{1cm}
    \large \textbf{Звіт про виконання лабораторної роботи №3}\\
    \large з курсу ``Advanced Computer Vision 2026''\\
    \vfill
    \Large \textbf{Тема: Створення пошукової системи по відео на основі моделі CLIP та кластеризація контенту}\\
    \vfill
    \large Виконав: студент групи ММШІ-2\\
    \large \textbf{Коломієць Микола}\\
    \vfill
    \large Дата: \today
\end{titlepage}

\section{Мета роботи}
Створення прототипу відео-пошукової системи, яка здатна приймати текстовий запит і повертати найбільш релевантні відеоматеріали. Додатковою метою є проведення автоматичної кластеризації відео-даних для групування контенту за тематиками.

\section{Використані технології та методи}
\begin{itemize}
    \item \textbf{Модель:} CLIP (ViT-B/32) від OpenAI для отримання векторних представлень (embeddings) тексту та зображень.
    \item \textbf{Обробка відео:} Фреймворк OpenCV для вибірки кадрів (sampling) з інтервалом в 1 секунду.
    \item \textbf{Агрегація:} Метод Mean Pooling для обчислення фінального дескриптора відео на основі ембеддінгів окремих кадрів.
    \item \textbf{Кластеризація:} Алгоритми K-Means та DBSCAN (метрика - cosine similarity).
\end{itemize}

\section{Опис експериментів та результатів}

\subsection{Пошук за текстовим запитом}
Для тестування було використано датасет, що складається з відео категорій ``Sports'' та ``Gaming''.
\begin{itemize}
    \item \textbf{Результат:} Система успішно ідентифікує контент. При запиті, що стосувався ігрової тематики, модель точно виділила фрагменти гри \textbf{Assassin's Creed} серед загального масиву даних.
    \item \textbf{Ранжування:} Косинусна близькість дозволила побудувати коректний топ результатів, де спортивний та ігровий контент чітко розмежовуються.
\end{itemize}

\subsection{Кластеризація (K-Means)}
При використанні K-Means з параметром $k=2$ (або більше), алгоритм стабільно розділив дані на дві великі групи:
\begin{enumerate}
    \item Спортивні змагання (футбол, атлетика тощо).
    \item Геймплей відеоігор.
\end{enumerate}

\subsection{Кластеризація (DBSCAN)}
Алгоритм DBSCAN показав більш деталізовану структуру:
\begin{itemize}
    \item Було виявлено значну кількість ``шуму'' (outliers), що відповідає відео з унікальними візуальними характеристиками.
    \item Алгоритм не лише розділив основні категорії, а й самостійно виділив специфічну підгрупу — \textbf{відео з Minecraft}, завдяки характерному кубічному стилю графіки, який утворює щільний кластер у векторному просторі CLIP.
\end{itemize}

\section{Висновки}
Розроблена система демонструє високу ефективність у задачах Zero-shot пошуку. Використання Mean Pooling для кадрів відео виявилося достатнім для точного тематичного пошуку та кластеризації. Модель CLIP успішно справляється з розпізнаванням складних візуальних патернів (наприклад, стилістика Assassin's Creed або Minecraft) без додаткового донавчання.

\end{document}